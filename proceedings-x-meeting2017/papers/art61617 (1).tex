
  \documentclass[twoside]{article}
  \usepackage[affil-it]{authblk}
  \usepackage{lipsum} % Package to generate dummy text throughout this template
  \usepackage{eurosym}
  \usepackage[sc]{mathpazo} % Use the Palatino font
  \usepackage[T1]{fontenc} % Use 8-bit encoding that has 256 glyphs
  \usepackage[utf8]{inputenc}
  \linespread{1.05} % Line spacing-Palatino needs more space between lines
  \usepackage{microtype} % Slightly tweak font spacing for aesthetics\[IndentingNewLine]
  \usepackage[hmarginratio=1:1,top=32mm,columnsep=20pt]{geometry} % Document margins
  \usepackage{multicol} % Used for the two-column layout of the document
  \usepackage[hang,small,labelfont=bf,up,textfont=it,up]{caption} % Custom captions under//above floats in tables or figures
  \usepackage{booktabs} % Horizontal rules in tables
  \usepackage{float} % Required for tables and figures in the multi-column environment-they need to be placed in specific locations with the[H] (e.g. \begin{table}[H])
  \usepackage{hyperref} % For hyperlinks in the PDF
  \usepackage{lettrine} % The lettrine is the first enlarged letter at the beginning of the text
  \usepackage{paralist} % Used for the compactitem environment which makes bullet points with less space between them
  \usepackage{abstract} % Allows abstract customization
  \renewcommand{\abstractnamefont}{\normalfont\bfseries} 
  %\renewcommand{\abstracttextfont}{\normalfont\small\itshape} % Set the abstract itself to small italic text\[IndentingNewLine]
  \usepackage{titlesec} % Allows customization of titles
  \renewcommand\thesection{\Roman{section}} % Roman numerals for the sections
  \renewcommand\thesubsection{\Roman{subsection}} % Roman numerals for subsections
  \titleformat{\section}[block]{\large\scshape\centering}{\thesection.}{1em}{} % Change the look of the section titles
  \titleformat{\subsection}[block]{\large}{\thesubsection.}{1em}{} % Change the look of the section titles
  \usepackage{fancyhdr} % Headers and footers
  \pagestyle{fancy} % All pages have headers and footers
  \fancyhead{} % Blank out the default header
  \fancyfoot{} % Blank out the default footer
  \fancyhead[C]{X-meeting $\bullet$ November 2017 $\bullet$ S\~ao Pedro} % Custom header text
  \fancyfoot[RO,LE]{} % Custom footer text
  %----------------------------------------------------------------------------------------
  % TITLE SECTION
  %---------------------------------------------------------------------------------------- 
 
 \title{\vspace{-15mm}\fontsize{24pt}{10pt}\selectfont\textbf{ The Pan-Genome of Treponema pallidum Reveals Differences in Genome Plasticity between the subspecies }} % Article title
  
  
  \author{ Arun Kumar Jaiswal$^{1}$, Sandeep Tiwari$^{2}$, Syed Babar Jamal Bacha$^{2}$, Vasco a de C Azevedo$^{3}$, Siomar de Castro Soares$^{4}$, }
  
  \affil{ 1 Institute of Biological Science, Federal University of Minas Gerais, Belo Horizonte;	Department of Immunology, Microbiology and Parasitology, Institute of Biological Sciences and Natural Sciences, Federal University of Triângulo Mineiro

2 Institute of Biological Science, Federal University of Minas Gerais, Belo Horizonte

3 Federal University of Minas Gerais

4 Department of Immunology, Microbiology and Parasitology, Institute of Biological Sciences and Natural Sciences, Federal University of Triângulo Mineiro 

 }
  \vspace{-5mm}
  \date{}
  
  %---------------------------------------------------------------------------------------- 
  
  \begin{document}
  
  
  \maketitle % Insert title
  
  
  \thispagestyle{fancy} % All pages have headers and footers
  %----------------------------------------------------------------------------------------  
  % ABSTRACT
  
  %----------------------------------------------------------------------------------------  
  
  \begin{abstract}
  Spirochetal organisms of the Treponema species are responsible for causing Treponematoses. Pathogenic treponemes cause multi-stage infections like endemic syphilis, venereal syphilis, yaws and pinta. Out of these four lethal diseases, venereal syphilis is transmitted by sexual contact; the other three diseases are transmitted by close personal contact. Treponema pallidum subspecies pallidum is Gram-negative, motile, spirochete pathogen that cause syphilis in human. Syphilis is a multistage infectious disease that can be communicated by sexual contact. The current worldwide prevalence of syphilis emphasizes the need for continued preventive measures and strategies. Unfortunately, effective measures are limited. The genome sequence of all 49 T. pallidum strains available from NCBI, isolated from different hosts and countries, were comparatively analysed using pan-genomic strategy. Phylogenomic, pan-genome, core genome and singleton analyses disclosed the close connection among all strains of the pathogen Treponema pallidum. The pan-genomic analysis showed that all the strains are highly clonal. Furthermore, the genome plasticity analysis among the subspecies T pallidum subsp pallidum, T. pallidum subsp endemicum and T. pallidum subsp pertenue revealed differences in the pathogenicity island (PAIs) and genomic island (GIs) repertoire. We found 4 pathogenicity island (PAIs) and 8 genomic island (GIs) in subsp pallidum, whereas subsp endemicum has 3 PAIs and 7 GIs and subsp pertenue harbour3 PAIs and 8 GIs. The differences observed in genome plasticity among sub species can be useful for further characterization oftheir epidemic behaviour.
  
  Funding: TWAS.CNPq and CAPES \\ 
  \end{abstract}
  \end{document} 