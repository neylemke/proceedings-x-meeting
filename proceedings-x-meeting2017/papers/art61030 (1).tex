
  \documentclass[twoside]{article}
  \usepackage[affil-it]{authblk}
  \usepackage{lipsum} % Package to generate dummy text throughout this template
  \usepackage{eurosym}
  \usepackage[sc]{mathpazo} % Use the Palatino font
  \usepackage[T1]{fontenc} % Use 8-bit encoding that has 256 glyphs
  \usepackage[utf8]{inputenc}
  \linespread{1.05} % Line spacing-Palatino needs more space between lines
  \usepackage{microtype} % Slightly tweak font spacing for aesthetics\[IndentingNewLine]
  \usepackage[hmarginratio=1:1,top=32mm,columnsep=20pt]{geometry} % Document margins
  \usepackage{multicol} % Used for the two-column layout of the document
  \usepackage[hang,small,labelfont=bf,up,textfont=it,up]{caption} % Custom captions under//above floats in tables or figures
  \usepackage{booktabs} % Horizontal rules in tables
  \usepackage{float} % Required for tables and figures in the multi-column environment-they need to be placed in specific locations with the[H] (e.g. \begin{table}[H])
  \usepackage{hyperref} % For hyperlinks in the PDF
  \usepackage{lettrine} % The lettrine is the first enlarged letter at the beginning of the text
  \usepackage{paralist} % Used for the compactitem environment which makes bullet points with less space between them
  \usepackage{abstract} % Allows abstract customization
  \renewcommand{\abstractnamefont}{\normalfont\bfseries} 
  %\renewcommand{\abstracttextfont}{\normalfont\small\itshape} % Set the abstract itself to small italic text\[IndentingNewLine]
  \usepackage{titlesec} % Allows customization of titles
  \renewcommand\thesection{\Roman{section}} % Roman numerals for the sections
  \renewcommand\thesubsection{\Roman{subsection}} % Roman numerals for subsections
  \titleformat{\section}[block]{\large\scshape\centering}{\thesection.}{1em}{} % Change the look of the section titles
  \titleformat{\subsection}[block]{\large}{\thesubsection.}{1em}{} % Change the look of the section titles
  \usepackage{fancyhdr} % Headers and footers
  \pagestyle{fancy} % All pages have headers and footers
  \fancyhead{} % Blank out the default header
  \fancyfoot{} % Blank out the default footer
  \fancyhead[C]{X-meeting $\bullet$ November 2017 $\bullet$ S\~ao Pedro} % Custom header text
  \fancyfoot[RO,LE]{} % Custom footer text
  %----------------------------------------------------------------------------------------
  % TITLE SECTION
  %---------------------------------------------------------------------------------------- 
 
 \title{\vspace{-15mm}\fontsize{24pt}{10pt}\selectfont\textbf{ New approach for genomic comparison of invasive and non-invasive strains of Streptococcus pyogenes }} % Article title
  
  
  \author{ Suzane de Andrade Barboza$^{1}$, Caio Rafael do Nascimento Santiago$^{1}$, Luciano Antonio Digiampietri$^{1}$, }
  
  \affil{ 1 Universidade de São Paulo

 }
  \vspace{-5mm}
  \date{}
  
  %---------------------------------------------------------------------------------------- 
  
  \begin{document}
  
  
  \maketitle % Insert title
  
  
  \thispagestyle{fancy} % All pages have headers and footers
  %----------------------------------------------------------------------------------------  
  % ABSTRACT
  
  %----------------------------------------------------------------------------------------  
  
  \begin{abstract}
  Streptococcus pyogenes or Group A streptococcal (GAS) is a uniquely human Gram-positive pathogen related to a wide range of invasive and non-invasive diseases, having the fourth highest mortality rate among bacterial pathogens. The diversity of clinical outcomes of these infections can be explained by the acquisition of exogenous genetic material, mostly composed of virulence factors such as adhesins or phage toxins. One of the main virulence factors is M protein, which hypervariable region is used for GAS classification. Molecular epidemiology studies showed a genotype M/pathogenicity relation, which is being intensively investigated by genomic comparisons. However, little is known about different invasive levels observed within strains sharing the same genotype. This lack of information occurs due to two main reasons: (1) recent studies limit their comparisons by genotype or pathology analysis, disregarding non-invasive strains, and (2) software limitations concerning closely related genomes comparisons, which include difficulties in performing global alignments considering large genome rearrangements and the identification of strains’ exclusive genes. In order to overcome these difficulties, two main strategies have been used in the comparison of 55 GAS genomes (28 genomes from invasive strains, 25 from non-invasive strains and 2 from isolates with unknown invasive profile): phylogenetic and gene network analysis, based on the identification of homologous genes. After performing local alignments with all genes of all genomes, homology relations were defined considering seven defined parameters: minimum identity percentage, minimum alignment percentage, minimum alignment length, maximum number of mismatched positions, maximum number of gap positions, maximum e-value, and minimum bit-score. The resulting graph is a gene network representation, where each homologous gene group is a cluster composed of nodes representing the genes of the genomes. Each genome is represented by a color, which allow us to identify gene sets exclusively found on invasive strains or strains related to a certain disease. A distance matrix of the genomes has then been calculated based on presence or absence of genes for each group of genes created previously, and a cladogram (representing the phylogenetic relationships) of all genomes was constructed, grouping strains with similar gene composition. Relating this information with the disease/virulence profile, we aim to better understand the relation between GAS genotypes and gene acquisition. These graphical representations will accelerate the identification of the virulence factors that could explain certain isolates’ invasiveness and alternative genes for the production of an anti-streptococcal vaccine.
  
  Funding: Capes \\ 
  \end{abstract}
  \end{document} 