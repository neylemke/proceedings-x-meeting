
  \documentclass[twoside]{article}
  \usepackage[affil-it]{authblk}
  \usepackage{lipsum} % Package to generate dummy text throughout this template
  \usepackage{eurosym}
  \usepackage[sc]{mathpazo} % Use the Palatino font
  \usepackage[T1]{fontenc} % Use 8-bit encoding that has 256 glyphs
  \usepackage[utf8]{inputenc}
  \linespread{1.05} % Line spacing-Palatino needs more space between lines
  \usepackage{microtype} % Slightly tweak font spacing for aesthetics\[IndentingNewLine]
  \usepackage[hmarginratio=1:1,top=32mm,columnsep=20pt]{geometry} % Document margins
  \usepackage{multicol} % Used for the two-column layout of the document
  \usepackage[hang,small,labelfont=bf,up,textfont=it,up]{caption} % Custom captions under//above floats in tables or figures
  \usepackage{booktabs} % Horizontal rules in tables
  \usepackage{float} % Required for tables and figures in the multi-column environment-they need to be placed in specific locations with the[H] (e.g. \begin{table}[H])
  \usepackage{hyperref} % For hyperlinks in the PDF
  \usepackage{lettrine} % The lettrine is the first enlarged letter at the beginning of the text
  \usepackage{paralist} % Used for the compactitem environment which makes bullet points with less space between them
  \usepackage{abstract} % Allows abstract customization
  \renewcommand{\abstractnamefont}{\normalfont\bfseries} 
  %\renewcommand{\abstracttextfont}{\normalfont\small\itshape} % Set the abstract itself to small italic text\[IndentingNewLine]
  \usepackage{titlesec} % Allows customization of titles
  \renewcommand\thesection{\Roman{section}} % Roman numerals for the sections
  \renewcommand\thesubsection{\Roman{subsection}} % Roman numerals for subsections
  \titleformat{\section}[block]{\large\scshape\centering}{\thesection.}{1em}{} % Change the look of the section titles
  \titleformat{\subsection}[block]{\large}{\thesubsection.}{1em}{} % Change the look of the section titles
  \usepackage{fancyhdr} % Headers and footers
  \pagestyle{fancy} % All pages have headers and footers
  \fancyhead{} % Blank out the default header
  \fancyfoot{} % Blank out the default footer
  \fancyhead[C]{X-meeting $\bullet$ November 2017 $\bullet$ S\~ao Pedro} % Custom header text
  \fancyfoot[RO,LE]{} % Custom footer text
  %----------------------------------------------------------------------------------------
  % TITLE SECTION
  %---------------------------------------------------------------------------------------- 
 
 \title{\vspace{-15mm}\fontsize{24pt}{10pt}\selectfont\textbf{ A global feature selection algorithm for the model selection step in the identification of cell signaling networks }} % Article title
  
  
  \author{ Gustavo Estrela de Matos$^{1}$, Lulu Wu$^{2}$, Vincent Noel$^{3}$, Marco Dimas Gubitoso$^{4}$, Carlos Eduardo Ferreira$^{4}$, Junior Barrera$^{4}$, Hugo A. Armelin$^{3}$, Marcelo S. Reis$^{3}$, }
  
  \affil{ 1 Center of Toxins, Immune-response and Cell Signaling, Instituto Butantan, Instituto de Matemática e Estatística

2 Center of Toxins, Immune-response and Cell Signaling, Instituto Butantan

3 Instituto Butantan

4 Instituto de Matemática e Estatística, Universidade de São Paulo

 }
  \vspace{-5mm}
  \date{}
  
  %---------------------------------------------------------------------------------------- 
  
  \begin{document}
  
  
  \maketitle % Insert title
  
  
  \thispagestyle{fancy} % All pages have headers and footers
  %----------------------------------------------------------------------------------------  
  % ABSTRACT
  
  %----------------------------------------------------------------------------------------  
  
  \begin{abstract}
  In the context of cell signaling network identification, model selection is the choice of a dynamic
model from a set of possibilities; the chosen model should be the most suitable one according to a
given cost function (e.g., curve-fitting optimization). If these possibilities are defined by differences
in the chemical species and/or reactions that compose each of them, then a feature selection
procedure could be carried out to accomplish the model selection. Recently, it was proposed a
method to carry out model selection through examination of interactome databases. However, such
databases typically yield huge search spaces during the feature selection procedure; hence, only a
greedy sequential approach could be explored so far. Therefore, there is a need for development of
efficient global feature selection methods to tackle this hard combinatorial optimization problem. In
this work, we introduce a new global feature selection method, which may be used to assist the
model selection step during the identification of cell signaling networks. This method, called
Parallelized U-Curve Search (PUCS), relies on the fact that the chain costs of the Boolean lattice
induced by the search space are decomposable in U-shaped curves; this latter phenomenon is due
the curse of dimensionality, that is, the impact the lack of samples brings to the cost function as the
number of considered features increases. To implement and evaluate the PUCS algorithm, we used
featsel, a framework for benchmarking of feature selection algorithms and cost functions. To
compute the cost function (i.e., the fitness of a candidate model), we are employing the Signaling
Network Simulator (SigNetSim), a tool for building, fitting, and analyzing mathematical models of
molecular signaling networks. Initial results with synthetic data showed that PUCS outperforms
golden standard algorithms in feature selection such as the Sequential Forward Floating Search
(SFFS). Currently, we are applying PUCS into the model selection of real-data signaling networks
extracted from the Kyoto Encyclopedia of Genes and Genomes (KEGG) interactome database. We
expect that this new feature selection method will become a critical asset to identify fully predictive
dynamic models, which in turn will help researchers to unveil intricate molecular mechanisms
underlying cell phenotype changes due extracellular stimuli.
  
  Funding: CAPES, CNPq and grants \#2013/07467-1 and \#2016/25959-7, S\~ao Paulo Research Foundation (FAPESP) \\ 
  \end{abstract}
  \end{document} 