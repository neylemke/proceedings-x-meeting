
  \documentclass[twoside]{article}
  \usepackage[affil-it]{authblk}
  \usepackage{lipsum} % Package to generate dummy text throughout this template
  \usepackage{eurosym}
  \usepackage[sc]{mathpazo} % Use the Palatino font
  \usepackage[T1]{fontenc} % Use 8-bit encoding that has 256 glyphs
  \usepackage[utf8]{inputenc}
  \linespread{1.05} % Line spacing-Palatino needs more space between lines
  \usepackage{microtype} % Slightly tweak font spacing for aesthetics\[IndentingNewLine]
  \usepackage[hmarginratio=1:1,top=32mm,columnsep=20pt]{geometry} % Document margins
  \usepackage{multicol} % Used for the two-column layout of the document
  \usepackage[hang,small,labelfont=bf,up,textfont=it,up]{caption} % Custom captions under//above floats in tables or figures
  \usepackage{booktabs} % Horizontal rules in tables
  \usepackage{float} % Required for tables and figures in the multi-column environment-they need to be placed in specific locations with the[H] (e.g. \begin{table}[H])
  \usepackage{hyperref} % For hyperlinks in the PDF
  \usepackage{lettrine} % The lettrine is the first enlarged letter at the beginning of the text
  \usepackage{paralist} % Used for the compactitem environment which makes bullet points with less space between them
  \usepackage{abstract} % Allows abstract customization
  \renewcommand{\abstractnamefont}{\normalfont\bfseries} 
  %\renewcommand{\abstracttextfont}{\normalfont\small\itshape} % Set the abstract itself to small italic text\[IndentingNewLine]
  \usepackage{titlesec} % Allows customization of titles
  \renewcommand\thesection{\Roman{section}} % Roman numerals for the sections
  \renewcommand\thesubsection{\Roman{subsection}} % Roman numerals for subsections
  \titleformat{\section}[block]{\large\scshape\centering}{\thesection.}{1em}{} % Change the look of the section titles
  \titleformat{\subsection}[block]{\large}{\thesubsection.}{1em}{} % Change the look of the section titles
  \usepackage{fancyhdr} % Headers and footers
  \pagestyle{fancy} % All pages have headers and footers
  \fancyhead{} % Blank out the default header
  \fancyfoot{} % Blank out the default footer
  \fancyhead[C]{X-meeting $\bullet$ November 2017 $\bullet$ S\~ao Pedro} % Custom header text
  \fancyfoot[RO,LE]{} % Custom footer text
  %----------------------------------------------------------------------------------------
  % TITLE SECTION
  %---------------------------------------------------------------------------------------- 
 
 \title{\vspace{-15mm}\fontsize{24pt}{10pt}\selectfont\textbf{ Analysis of genomic islands of virulence and pathogenicity in Xanthomonas campestris }} % Article title
  
  
  \author{ Juan Carlos Ariute$^{1}$, João Pacifico Bezerra Neto$^{2}$, Ana Maria Benko-Iseppon$^{1}$, Flavia Figueira Aburjaile$^{2}$, }
  
  \affil{ 1 Federal University of Pernambuco Center of Biological Sciences Genetics Dept.

2 Federal University of Pernambuco Center of Biological Sciences  Genetics Dept.

 }
  \vspace{-5mm}
  \date{}
  
  %---------------------------------------------------------------------------------------- 
  
  \begin{document}
  
  
  \maketitle % Insert title
  
  
  \thispagestyle{fancy} % All pages have headers and footers
  %----------------------------------------------------------------------------------------  
  % ABSTRACT
  
  %----------------------------------------------------------------------------------------  
  
  \begin{abstract}
  Brazilian viticulture activity has remarkably increased in the northeastern region of the
country over the last years. However, significant losses have occurred due to the local
climate rough conditions that raise susceptibility of grape vine species such as Vitis
vinifera to bacterial infections. In this context, Xanthomonas campestris pv. viticola, a
Gram negative aerobic pathogenic bacteria, is associated with bacterial canker and
many other phytopathologies, due to factors such as xanthomonadin and xantham gum
production. Since it was first isolated from an Indian viticulture in the 70’s, extensive
studies have tried to elucidate how to control X. campestris pv. viticola infections in
grape vine. Nevertheless, there is still a shortage of information regarding the molecular
mechanisms and the acquisition of virulence genes in this organism. Therefore, we
believe that identifying and exploring genomic content from Xanthomonas campestris
strains would improve knowledge concerning its pathogenicity and help develop future
approaches to control the disease. Pathogenicity islands (PAI) are described as a set of
virulent genes which have been horizontally transferred among Eubacteria, providing
the organism with the ability to cause pathologies. In this sense, we aimed to identify
genes that are present in PAIs from X. campestris pv. viticola. For automatic annotation
process, seven complete genome of X. campestris pv. campestris strains obtained from
National Center for Biotechnology Information (NCBI) genome database were submitted
on Rapid Annotation using Subsystem Technology (RAST). Afterwards, the annotated
genomes were analyzed on Seed viewer and Artemis. Finally, Genomic Island
Prediction Software (GIPSy) and Islandviewer4 were used to provide a better prediction
view and analysis of PAIs. We show that almost 45\% of genome can be annotated
using subsystems and that 77 genes are directly related to virulence, defense
mechanisms and disease activity. The proteins encoded by these 77 genes were
divided in two basic subcategories: antibacterial peptides and resistance to toxic
compounds and antibiotics related to copper homeostasis. In summary, we reveal that
the genomes of all seven Xanthomonas strains are very similar to each other but do
possess relevant differences in terms of genes encoding pathogenicity factors.
Moreover, there are high expectations for finding a resembling pattern on X. campestris
pv. viticola, making possible the development of genetic improvements of plants,
without the use of chemical agents.
  
  Funding: CAPES, FACEPE, CNPq \\ 
  \end{abstract}
  \end{document} 