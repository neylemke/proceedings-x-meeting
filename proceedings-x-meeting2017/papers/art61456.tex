
  \documentclass[twoside]{article}
  \usepackage[affil-it]{authblk}
  \usepackage{lipsum} % Package to generate dummy text throughout this template
  \usepackage{eurosym}
  \usepackage[sc]{mathpazo} % Use the Palatino font
  \usepackage[T1]{fontenc} % Use 8-bit encoding that has 256 glyphs
  \usepackage[utf8]{inputenc}
  \linespread{1.05} % Line spacing-Palatino needs more space between lines
  \usepackage{microtype} % Slightly tweak font spacing for aesthetics\[IndentingNewLine]
  \usepackage[hmarginratio=1:1,top=32mm,columnsep=20pt]{geometry} % Document margins
  \usepackage{multicol} % Used for the two-column layout of the document
  \usepackage[hang,small,labelfont=bf,up,textfont=it,up]{caption} % Custom captions under//above floats in tables or figures
  \usepackage{booktabs} % Horizontal rules in tables
  \usepackage{float} % Required for tables and figures in the multi-column environment-they need to be placed in specific locations with the[H] (e.g. \begin{table}[H])
  \usepackage{hyperref} % For hyperlinks in the PDF
  \usepackage{lettrine} % The lettrine is the first enlarged letter at the beginning of the text
  \usepackage{paralist} % Used for the compactitem environment which makes bullet points with less space between them
  \usepackage{abstract} % Allows abstract customization
  \renewcommand{\abstractnamefont}{\normalfont\bfseries} 
  %\renewcommand{\abstracttextfont}{\normalfont\small\itshape} % Set the abstract itself to small italic text\[IndentingNewLine]
  \usepackage{titlesec} % Allows customization of titles
  \renewcommand\thesection{\Roman{section}} % Roman numerals for the sections
  \renewcommand\thesubsection{\Roman{subsection}} % Roman numerals for subsections
  \titleformat{\section}[block]{\large\scshape\centering}{\thesection.}{1em}{} % Change the look of the section titles
  \titleformat{\subsection}[block]{\large}{\thesubsection.}{1em}{} % Change the look of the section titles
  \usepackage{fancyhdr} % Headers and footers
  \pagestyle{fancy} % All pages have headers and footers
  \fancyhead{} % Blank out the default header
  \fancyfoot{} % Blank out the default footer
  \fancyhead[C]{X-meeting $\bullet$ November 2017 $\bullet$ S\~ao Pedro} % Custom header text
  \fancyfoot[RO,LE]{} % Custom footer text
  %----------------------------------------------------------------------------------------
  % TITLE SECTION
  %---------------------------------------------------------------------------------------- 
 
 \title{\vspace{-15mm}\fontsize{24pt}{10pt}\selectfont\textbf{ Alignment of the SSR microsatellite markers sequences with the cassava genome (Manihot esculenta) }} % Article title
  
  
  \author{ Vanesca Priscila Camargo Rocha$^{1}$, Daniel Longhi Fernandes Pedro$^{2}$, }
  
  \affil{ 1 Universidade Tecnológica Federal do Paraná

2 UTFPR - PPGBIOINFO

 }
  \vspace{-5mm}
  \date{}
  
  %---------------------------------------------------------------------------------------- 
  
  \begin{document}
  
  
  \maketitle % Insert title
  
  
  \thispagestyle{fancy} % All pages have headers and footers
  %----------------------------------------------------------------------------------------  
  % ABSTRACT
  
  %----------------------------------------------------------------------------------------  
  
  \begin{abstract}
  The species Manihot esculenta Crantz is cultivated in 105 countries, mainly in places located at latitude 30 $^o$ N and 30 $^o$ S (Nassar and Ortiz 2007), presenting wide adaptability to dry environments and to soils with low fertility, being able to withstand drought for periods of time leaf canopy and loss of water due to transpiration (El-Sharkawy 2004). The main product of cassava are the tuberous roots that are intended for human consumption, animal and in the industry of starch and flour processing. The objective of the present study was to perform the alignment of the microsatellite primers on the cassava genome and to verify in which chromosomes and ORFs are located. A comparative global alignment analysis of the 25 GA and SSRY forward primers in the cassava genome made available for the public use of these data in Phytozome (Manihot esculenta, version 6.1) was carried out and later the coding regions were applied in the tool provided by N.C.B.I. (National Center for Biotechnology Information) called Protein BLAST. Of the 25 loci analyzed, only five loci (SSRY 85, SSRY 65, GA 134, GA 140 and GA 161) were not located in the genome. The hypothesis for this case would be because they are in regions that do not encode proteins. Although, an alignment of the SSRY 85 loci was performed on the Arabidopsis thaliana genome and the same sequence was localized to seven possible genes in the genome. Some loci obtained identity below 28\% as is the case of locus GA 12 (CHR 16, ORF 1); Locus SSRY 100 (CHR9, ORF 3); Locus SSRY 19 (CHR 1, ORF 3 and CHR 10, ORF 1). The GA 136 locus obtained 85\% identity that is linked to expression for protein kinase, SSRY 19 with 81\% expressed the beta-amylase protein, SSRY 27 with 94\% identity for ethylene receptor expression. Microsatellite molecular markers can be used to identify genes that correspond to proteins.
  
  Funding: Funda\c{c}\~ao Arauc\'aria \\ 
  \end{abstract}
  \end{document} 