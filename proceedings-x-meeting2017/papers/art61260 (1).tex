
  \documentclass[twoside]{article}
  \usepackage[affil-it]{authblk}
  \usepackage{lipsum} % Package to generate dummy text throughout this template
  \usepackage{eurosym}
  \usepackage[sc]{mathpazo} % Use the Palatino font
  \usepackage[T1]{fontenc} % Use 8-bit encoding that has 256 glyphs
  \usepackage[utf8]{inputenc}
  \linespread{1.05} % Line spacing-Palatino needs more space between lines
  \usepackage{microtype} % Slightly tweak font spacing for aesthetics\[IndentingNewLine]
  \usepackage[hmarginratio=1:1,top=32mm,columnsep=20pt]{geometry} % Document margins
  \usepackage{multicol} % Used for the two-column layout of the document
  \usepackage[hang,small,labelfont=bf,up,textfont=it,up]{caption} % Custom captions under//above floats in tables or figures
  \usepackage{booktabs} % Horizontal rules in tables
  \usepackage{float} % Required for tables and figures in the multi-column environment-they need to be placed in specific locations with the[H] (e.g. \begin{table}[H])
  \usepackage{hyperref} % For hyperlinks in the PDF
  \usepackage{lettrine} % The lettrine is the first enlarged letter at the beginning of the text
  \usepackage{paralist} % Used for the compactitem environment which makes bullet points with less space between them
  \usepackage{abstract} % Allows abstract customization
  \renewcommand{\abstractnamefont}{\normalfont\bfseries} 
  %\renewcommand{\abstracttextfont}{\normalfont\small\itshape} % Set the abstract itself to small italic text\[IndentingNewLine]
  \usepackage{titlesec} % Allows customization of titles
  \renewcommand\thesection{\Roman{section}} % Roman numerals for the sections
  \renewcommand\thesubsection{\Roman{subsection}} % Roman numerals for subsections
  \titleformat{\section}[block]{\large\scshape\centering}{\thesection.}{1em}{} % Change the look of the section titles
  \titleformat{\subsection}[block]{\large}{\thesubsection.}{1em}{} % Change the look of the section titles
  \usepackage{fancyhdr} % Headers and footers
  \pagestyle{fancy} % All pages have headers and footers
  \fancyhead{} % Blank out the default header
  \fancyfoot{} % Blank out the default footer
  \fancyhead[C]{X-meeting $\bullet$ November 2017 $\bullet$ S\~ao Pedro} % Custom header text
  \fancyfoot[RO,LE]{} % Custom footer text
  %----------------------------------------------------------------------------------------
  % TITLE SECTION
  %---------------------------------------------------------------------------------------- 
 
 \title{\vspace{-15mm}\fontsize{24pt}{10pt}\selectfont\textbf{ CNV calling and its characterization in the Brazilian population }} % Article title
  
  
  \author{ Ana Claudia Martins Ciconelle$^{1}$, Júlia Maria Pavan Soler$^{1}$, }
  
  \affil{ 1 Instituto de Matemática e Estatística - IME/USP

 }
  \vspace{-5mm}
  \date{}
  
  %---------------------------------------------------------------------------------------- 
  
  \begin{document}
  
  
  \maketitle % Insert title
  
  
  \thispagestyle{fancy} % All pages have headers and footers
  %----------------------------------------------------------------------------------------  
  % ABSTRACT
  
  %----------------------------------------------------------------------------------------  
  
  \begin{abstract}
  A copy number variation (CNV) occurs when the number of copies of a particular
region of the DNA differs from two in autosomes or one/two in allosomes and has an
important role in the genetic variability in humans. The effects of CNVs to human
diseases are not yet known, although several diseases have been associated to this kind
of polymorphism, such as uric acid and nervous system disorders. 
Motivated by the unknown influence of CNVs on anthropometric measurements and
cardiovascular phenotypes and in collaboration with the Laboratory of Genetics and
Molecular Cardiology at the Heart Institute/InCor-FMUSP, the primary aim in this
project is to estimate the CNV from SNP array platforms and understand its
transmission rate in family data and its association with complex phenotypes.  This
project also aims to understand the CNV distribution in Brazilian population and
between family members.
A pipeline was proposed for CNV calling from SNP Array data by reviewing softwares
and packages that are available in the literature. Using the database from the Baependi
Heart project, we analyzed the genotype and phenotype data from 80 families to identify
the CNVs and to understand their association with height. From the genetic data, the
CNVs were estimated from a combination of statistical techniques and algorithms
including quantile normalization, classification methods for genotyping of SNPs and
hidden Markov chains. After an exploratory data analysis, polygenic linear mixed
models were used to estimate the association of CNVs with the chosen phenotypes. 
Our results suggest that the Brazilian population have a similar number of CNV per
person (around 55 CNVs) in comparison with other populations. However, from the
total of 64.107 CNVs identified, 147 CNVs are common among our samples, but rare in
worldwide populations, one example being a CNV in the NEGR1 gene, which is present
in 89\% of our sample. Based on family data, the intraclass correlation coefficient for
CNVs was estimated between 30\% to 60\% showing a high similarity on values from the
same family. Association analysis between CNVs and different phenotypes are being
performed.
  
  Funding: CNPq, CAPES \\ 
  \end{abstract}
  \end{document} 