
  \documentclass[twoside]{article}
  \usepackage[affil-it]{authblk}
  \usepackage{lipsum} % Package to generate dummy text throughout this template
  \usepackage{eurosym}
  \usepackage[sc]{mathpazo} % Use the Palatino font
  \usepackage[T1]{fontenc} % Use 8-bit encoding that has 256 glyphs
  \usepackage[utf8]{inputenc}
  \linespread{1.05} % Line spacing-Palatino needs more space between lines
  \usepackage{microtype} % Slightly tweak font spacing for aesthetics\[IndentingNewLine]
  \usepackage[hmarginratio=1:1,top=32mm,columnsep=20pt]{geometry} % Document margins
  \usepackage{multicol} % Used for the two-column layout of the document
  \usepackage[hang,small,labelfont=bf,up,textfont=it,up]{caption} % Custom captions under//above floats in tables or figures
  \usepackage{booktabs} % Horizontal rules in tables
  \usepackage{float} % Required for tables and figures in the multi-column environment-they need to be placed in specific locations with the[H] (e.g. \begin{table}[H])
  \usepackage{hyperref} % For hyperlinks in the PDF
  \usepackage{lettrine} % The lettrine is the first enlarged letter at the beginning of the text
  \usepackage{paralist} % Used for the compactitem environment which makes bullet points with less space between them
  \usepackage{abstract} % Allows abstract customization
  \renewcommand{\abstractnamefont}{\normalfont\bfseries} 
  %\renewcommand{\abstracttextfont}{\normalfont\small\itshape} % Set the abstract itself to small italic text\[IndentingNewLine]
  \usepackage{titlesec} % Allows customization of titles
  \renewcommand\thesection{\Roman{section}} % Roman numerals for the sections
  \renewcommand\thesubsection{\Roman{subsection}} % Roman numerals for subsections
  \titleformat{\section}[block]{\large\scshape\centering}{\thesection.}{1em}{} % Change the look of the section titles
  \titleformat{\subsection}[block]{\large}{\thesubsection.}{1em}{} % Change the look of the section titles
  \usepackage{fancyhdr} % Headers and footers
  \pagestyle{fancy} % All pages have headers and footers
  \fancyhead{} % Blank out the default header
  \fancyfoot{} % Blank out the default footer
  \fancyhead[C]{X-meeting $\bullet$ November 2017 $\bullet$ S\~ao Pedro} % Custom header text
  \fancyfoot[RO,LE]{} % Custom footer text
  %----------------------------------------------------------------------------------------
  % TITLE SECTION
  %---------------------------------------------------------------------------------------- 
 
 \title{\vspace{-15mm}\fontsize{24pt}{10pt}\selectfont\textbf{ Genomic analysis of Corynebacterium pseudotuberculosis strain 262 }} % Article title
  
  
  \author{ Raquel Enma Hurtado Castillo$^{1}$, Marcus Vinicius Canário Viana$^{1}$, Anne Cybelle Pinto Gomide$^{1}$, Vasco A. de C. Azevedo$^{1}$, Rommel Thiago Jucá Ramos$^{2}$, Artur Silva$^{3}$, }
  
  \affil{ 1 Federal University of Minas Gerais

2 Univsersidade Federal do Para

3 Federal University of Pará

 }
  \vspace{-5mm}
  \date{}
  
  %---------------------------------------------------------------------------------------- 
  
  \begin{document}
  
  
  \maketitle % Insert title
  
  
  \thispagestyle{fancy} % All pages have headers and footers
  %----------------------------------------------------------------------------------------  
  % ABSTRACT
  
  %----------------------------------------------------------------------------------------  
  
  \begin{abstract}
  Corynebacterium pseudotuberculosis is a Gram-positive and facultative intracellular pathogen, causing important economic losses mainly in the ruminant production. The biovar ovis is nitrate negative and causes caseous lymphadenitis in sheep and goats, while biovar equi is nitrate positive and causes ulcerative lymphangitis, mastitis, and oedematous skin disease in a wide range of hosts. C. pseudotuberculosis 262 is an equi biovar strain isolated from cow milk. Genomic and phylogenomic analysis of C. pseudotuberculosis strains have been shown this strain as the most external of equi genomes and the closest one to ovis. In order to better characterize its genomic features, we present here a comparative genomic analysis between strain 262 and other 52 strains of C. pseudotuberculosis. A phylogenetic analysis based on a gene presence-absence matrix among strains of C. pseudotuberculosis does not cluster strain 262 in any of the two different clusters (biovars equi and ovis). Accessory genes shared between strain 262 and equi strains were predicted, such as a toxin and antirepressor, immunity-specific protein, CAAX protease self-immunity, serine hydrolase and superoxide dismutase. Accessory genes among 262 and ovis strains are genes related to ABC transporter protein, spermidine synthase, secreted protein and surface-anchored membrane protein. Sixteen genomic islands were predicted. Part of an island PiCp1 is a region of 10 Kb shared only with biovar equi, which presents CRISPR-associated protein. We also identified regions shared only with biovar ovis strains. In addition, strain 262 presented unique regions, containing 49 genes, such as MFS transporter, secreted protein, and hypothetical proteins, that could carry genes potentially associated with virulence. Finally, the pan-genomic analysis report accessory and unique genes that allow characterization of the strain 262. These findings enable us to better characterize the unique genomic features of strain 262 and generate new hypothesis to understand the differentiation of the C. pseudotuberculosis.
  
  Funding: CNPq, CAPES \\ 
  \end{abstract}
  \end{document} 