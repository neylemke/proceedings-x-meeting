
  \documentclass[twoside]{article}
  \usepackage[affil-it]{authblk}
  \usepackage{lipsum} % Package to generate dummy text throughout this template
  \usepackage{eurosym}
  \usepackage[sc]{mathpazo} % Use the Palatino font
  \usepackage[T1]{fontenc} % Use 8-bit encoding that has 256 glyphs
  \usepackage[utf8]{inputenc}
  \linespread{1.05} % Line spacing-Palatino needs more space between lines
  \usepackage{microtype} % Slightly tweak font spacing for aesthetics\[IndentingNewLine]
  \usepackage[hmarginratio=1:1,top=32mm,columnsep=20pt]{geometry} % Document margins
  \usepackage{multicol} % Used for the two-column layout of the document
  \usepackage[hang,small,labelfont=bf,up,textfont=it,up]{caption} % Custom captions under//above floats in tables or figures
  \usepackage{booktabs} % Horizontal rules in tables
  \usepackage{float} % Required for tables and figures in the multi-column environment-they need to be placed in specific locations with the[H] (e.g. \begin{table}[H])
  \usepackage{hyperref} % For hyperlinks in the PDF
  \usepackage{lettrine} % The lettrine is the first enlarged letter at the beginning of the text
  \usepackage{paralist} % Used for the compactitem environment which makes bullet points with less space between them
  \usepackage{abstract} % Allows abstract customization
  \renewcommand{\abstractnamefont}{\normalfont\bfseries} 
  %\renewcommand{\abstracttextfont}{\normalfont\small\itshape} % Set the abstract itself to small italic text\[IndentingNewLine]
  \usepackage{titlesec} % Allows customization of titles
  \renewcommand\thesection{\Roman{section}} % Roman numerals for the sections
  \renewcommand\thesubsection{\Roman{subsection}} % Roman numerals for subsections
  \titleformat{\section}[block]{\large\scshape\centering}{\thesection.}{1em}{} % Change the look of the section titles
  \titleformat{\subsection}[block]{\large}{\thesubsection.}{1em}{} % Change the look of the section titles
  \usepackage{fancyhdr} % Headers and footers
  \pagestyle{fancy} % All pages have headers and footers
  \fancyhead{} % Blank out the default header
  \fancyfoot{} % Blank out the default footer
  \fancyhead[C]{X-meeting $\bullet$ November 2017 $\bullet$ S\~ao Pedro} % Custom header text
  \fancyfoot[RO,LE]{} % Custom footer text
  %----------------------------------------------------------------------------------------
  % TITLE SECTION
  %---------------------------------------------------------------------------------------- 
 
 \title{\vspace{-15mm}\fontsize{24pt}{10pt}\selectfont\textbf{ Comparative genomics of six Pseudomonas phages isolated from composting }} % Article title
  
  
  \author{ Fernando Pacheco Nobre Rossi$^{1}$, Deyvid Amgarten$^{1}$, João Carlos Setubal$^{2}$, Aline Maria da Silva$^{2}$, }
  
  \affil{ 1 USP - Departamento de Quimica

2 USP

 }
  \vspace{-5mm}
  \date{}
  
  %---------------------------------------------------------------------------------------- 
  
  \begin{document}
  
  
  \maketitle % Insert title
  
  
  \thispagestyle{fancy} % All pages have headers and footers
  %----------------------------------------------------------------------------------------  
  % ABSTRACT
  
  %----------------------------------------------------------------------------------------  
  
  \begin{abstract}
  Bacteriophages (or simply phages) are viruses that infect bacterial cells and are the most abundant and, potentially, the most diverse biological entities on Earth. More than 1023 infections by phages are expected to occur every second. The dynamics of phage-host populations presents complex relationships and is thought to contribute to bacterial abundance and diversity as well as to environment homeostasis. In the billion years that phages co-existed with their bacterial hosts, phages have evolved highly diverse proteins that either inhibit or adapt bacterial metabolic processes to their own benefit. Since their discovery, in the early 20th century, phages and their proteins have been exploited as valuable molecular biology and biotechnology tools. Phages have been also considered potential antibacterial agents, and their use to reduce or eliminate bacterial infections is known as phage therapy. Phages might be a treatment option for?antibiotic resistant?bacteria. Phages that are lytic and that are not capable of displaying lysogeny are preferred for phage therapy purposes. In a previous work from our group, composting samples from the Sao Paulo Zoo Park were screened for phages infecting Pseudomonas aeruginosa PA14. Six phages were isolated and had their genome sequenced. One of them (ZC01) was shown to be from Siphoviridae Yu-A like genus and the other two (ZC03 and ZC08) were similar to each other and shown to be novel Podoviridae phages. All three phages are lytic and have the ability to degrade P. aeruginosa PA14 biofilm, and as such they can be promising candidates for antimicrobial application. In the present work, we extend this prior study by analyzing the three remaining phages (ZC04, ZC06 and ZC07). All three were predicted to belong to the Podoviridae family. Phylogenetic trees were generated based on multiple alignment of the terL marker gene using MAFFT, followed by alignment curation with GUIDANCE2 and maximum likelihood computation using RAxML. This analysis shows that phages ZC03, ZC08, ZC04, ZC06 and ZC07 are phylogenetically close. ZC06 and ZC07 are closer to each other than to others and their genomes have 99\% similarity. The genomes of ZC04 and ZC03 have 97\% similarity. Differences in these five genomes include INDELS that resulted in truncation of at least one CDS in ZC04. Other genomic differences that might have functional implications will be discussed.
  
  Funding: Funding for this research is provided by FAPESP and CAPES \\ 
  \end{abstract}
  \end{document} 